%==========================================
%
% Academic thesis template
%
% Based on KOMA-Script book class.
%
% This template is not officially endorsed 
% by any educational institution.
%
% Ben Swift   22/6/12
% benjamin.j.swift@gmail.com
% http://github.com/benswift/thesis-template
% 
% This template is in the public domain
% 
%==========================================
% preamble
\documentclass[12pt,xetex]{scrbook}
\KOMAoptions{%
  headings=normal,
  captions=rightbeside,
  bibliography=totoc,
  listof=totoc}

% fonts (need to have Linux Libertine .otf fonts installed)
\usepackage{libertineotf}
\usepackage{amsmath}
\usepackage{amsfonts}
\newcommand{\argmax}[1]{\underset{#1}{\operatorname{argmax}}}

% custom monospace font - I like Inconsolata
\usepackage{fontspec}
\setmonofont[Scale=MatchLowercase,Mapping=tex-text]{Inconsolata}

% tables
\usepackage{booktabs}
\usepackage{tabularx}
% if you want colour tables
% \usepackage{colortbl}

% useful for giving `named' colours
\usepackage[svgnames,hyperref]{xcolor}

% Put all your figures in a `figures' subdirectory
\DeclareGraphicsExtensions{.pdf,.png,.jpg}
\graphicspath{{./figures/}}

% use subcaption for multi-part figures
\usepackage[margin=10pt,labelfont=bf]{caption}
\usepackage[labelformat=simple]{subcaption}
\renewcommand\thesubfigure{(\alph{subfigure})}

% handy bits and pieces
\usepackage{metalogo} % provides xelatex logo
\usepackage{hologo} % provides xelatex logo
\usepackage{verbatim}

% double spacing
\usepackage{setspace}

% custom list environments - feel free to add your own
\usepackage{enumitem}

\newlist{transcriptlist}{description}{1}
\setlist[transcriptlist]{font=\sffamily\bfseries,
                              align=left,
                              leftmargin=1.6cm,
                              labelindent=\parindent, 
                              labelwidth=*}

\newenvironment{transcript}%
{\small\begin{transcriptlist}}%
{\end{transcriptlist}}

\newlist{headinglist}{description}{1}
\setlist[headinglist]{font=\sffamily\bfseries, 
                           leftmargin=0cm,
                           style=nextline}

% Uses biblatex for reference styling - see the biblatex documentation
\usepackage[%
    backend=bibtex,
style=verbose]{biblatex}
\bibliography{thesis}

% csquotes - for inline and block quoting
\usepackage[english=british,threshold=15,thresholdtype=words]{csquotes}
\SetCiteCommand{\parencite}

\newenvironment*{smallquote}
  {\quote\small}
  {\endquote}
\SetBlockEnvironment{smallquote}

% hyperref & bookmark
\usepackage[%
unicode=true,
hyperindex=true,
bookmarks=true,
pdftitle={Thesis Title},
pdfauthor={Thesis author},
colorlinks=true, % change to false for final printing
pdfborder=0,
allcolors=DarkBlue,
% plainpages=false,
pdfpagelabels,
hyperfootnotes=true]{hyperref}

\usepackage{bookmark}

\setcounter{tocdepth}{1} % only include `sections' in toc 

% glossaries package
\usepackage[%
acronym, 
nomain,
toc=true]{glossaries}

% for `clever' referencing - needs to be loaded after hyperref
\usepackage{cleveref}

% set up indexing macros
\makeindex
\makeglossaries

%============================================
%============================================
% document 

\begin{document}

% acronyms (uses glossaries package)
% add your own acronyms here, and then refer to them in the text as
% e.g. \gls{tla}

\newacronym{tla}{TLA}{Three Letter Acronym}

% other definitions

\renewcommand{\thepage}{\roman{page}}

%\newcommand{\figwidth}{4.05in}

%============================================
%============================================
% title page and dedication

\pagestyle{empty}

\title{Thesis Title}
% \subtitle{Thesis Subtitle}

\author{Thesis author}

\date{\today}

\publishers{%
  A thesis submitted for the degree of \\
  Doctor of Philosophy of \\
  The University of Wherever}

\uppertitleback{%
  \textbf{Institutional Address}\\

  Research School of Taxidermy\\
  Building 13\\
  University of Wherever\\
  \textsc{Some Country}\\
  \bigskip\\
  \textbf{Supervisory Panel}\\

  Prof. Alex Pingu  (Panel Chair)\\
  Research School of Taxidermy\\
  The University of Wherever\\

  Dr. John Oldmate\\
  School of Cooking\\
  The University of Wherever\\
  \bigskip\\
  Set in Linux Libertine with the help of {\KOMAScript} and
  \XeLaTeX.\\

  \copyright~\the\year. All rights reserved.}

\dedication{\emph{For my Ma and Pa}}

\pdfbookmark{Title Page}{Title Page}
\maketitle

%============================================
%============================================
% preliminaries

\frontmatter

% declaration

\cleardoublepage

\vspace*{0.4\textheight}

\begin{onehalfspace}
I hereby declare that the material presented here is the outcome
of the research project I have undertaken during my candidacy, that
I am the sole author unless otherwise indicated and that I have
fully documented the source of ideas, quotations or paraphrases
attributable to other authors.
\end{onehalfspace}

\vspace*{4cm}

\begin{flushright}
  \begin{minipage}{4cm}
    My name\\
    \today
  \end{minipage}
\end{flushright}

\cleardoublepage

\addchap{Publications}
\label{chap:pub}

Here's some stuff I wrote.

\cleardoublepage

\addchap{Acknowledgements}
\label{chap:ack}

Thanks so much, everyone.  You're all fantastic and I love youse all.

\cleardoublepage

\begin{onehalfspace}

\addchap{Abstract}

If I was forced to describe this thesis in less than 500 words, this
is how I'd do it\ldots

% tables of contents, figures, tables

\cleardoublepage
\pdfbookmark{Contents}{Contents}
\tableofcontents
\listoffigures
\listoftables

% make acronym list

\printglossary[type=\acronymtype,title=Abbreviations]

%============================================
%============================================
% main text

\mainmatter

\pagestyle{headings}

\setchapterpreamble[u]{%
  \dictum[\emph{The Carpenters}---We've only just begun]{We've only just begun\ldots}
  \bigskip}
\chapter{Introduction}
\label{chap:intro}

There are lots of good places to get help with writing LaTeX, so I
won't go into that here.\footcite[p.~433]{example1970}  
It is worth noting that the packages used in
this thesis template (which you can see if you read the `header' of
the \texttt{thesis.tex} file have pretty good documentation, so it
might be worth looking at that for clues on how to use the 
\texttt{glossaries}, and the \texttt{csquotes} package.
\footcite[p.~434]{example1970}

For figures, you can either use single figures (as in
\cref{fig:anchor}) or multi-part figures (as in \cref{fig:dog-figure})
using the \texttt{subcaption} package. Also, notice that the word
`figure' (or `table', or `equation') is automatically inserted by the
\texttt{cref} macro from the \texttt{cleveref} package. Isn't that
\emph{clever}.

\begin{figure}
  \centering
  \includegraphics[width=0.7\textwidth]{anchor}
  \caption{A picture of an anchor.  Shared by flickr user
    \texttt{phill.lister} under a Creative Commons attribution licence
     (CC BY 2.0).}
  \label{fig:anchor}
\end{figure}

\begin{figure}
  \centering
  \subcaptionbox{A dog.\label{fig:right-way-up-dog}}[\textwidth]%
  {\includegraphics[width=0.7\textwidth]{dog-1}}\\
  \vspace{1cm}
  \subcaptionbox{The same dog (presumably), but this time
    upside-down.  Crazy times.\label{fig:upside-down-dog}}[\textwidth]%
  {\includegraphics[width=0.7\textwidth]{dog-2}}%
  \caption{An example of a multi-part figure using the
    \texttt{subcaption} package. Pictures shared by flickr user
    \texttt{pamplemousse2007} under a Creative Commons attribution
    licence (CC BY 2.0).}
  \label{fig:dog-figure}
\end{figure}


%============================================
%============================================

% \setchapterpreamble[u]{%
%   \dictum[Dorothy, Wizard of Oz]{Toto, I've a feeling we're not in Kansas anymore.}
%   \bigskip}
\setchapterpreamble[u]{%
  \dictum[\emph{The Hollies}, He 'Aint Heavy]{%
    The road is long,\\
    with many a winding turn\\
    That leads us to who\\
    knows where, who knows where?}
  \bigskip}

\chapter{The second chapter}
\label{chap:second-chapter}

Now, all you have to do is write your thesis.  Piece of cake.

%============================================
%============================================

\chapter{Conclusion}
\label{chap:conclusion}

You probably should tie it all together here.

\section{Summary of contributions}
\label{sec:summary-of-contributions}

\begin{headinglist}
\item[Some bollocks]
  We did some bollocks, and it was pretty interesting.
\item[Some other bollocks] 
  Really, if you think about it, we did a whole other lot of bollocks
  as well.
\end{headinglist}

%============================================
%============================================
% end matter

\appendix

\chapter{An appendix}
\label{chap:an-appendix}

Put any appendices here---they are just like regular chapters, except
they follow the \texttt{\textbackslash{}appendix} directive.

\end{onehalfspace}

\backmatter

\bookmarksetup{startatroot}

\printbibliography[title=References,heading=bibintoc]

\end{document}
